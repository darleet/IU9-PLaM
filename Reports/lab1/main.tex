\documentclass[a4paper, 14pt]{extarticle}

% Поля
%----------------------
\usepackage{geometry}
\geometry{a4paper,left=2cm,right=1cm,
    top=2cm,bottom=2cm,bindingoffset=0cm}
%----------------------

% Russian-specific packages
%----------------------
\usepackage[T2A]{fontenc}
\usepackage[utf8]{inputenc}
\usepackage[english, main=russian]{babel}
%----------------------

\usepackage{textcomp}

% Красная строка
%----------------------
\usepackage{indentfirst}
%----------------------

% Graphics
%----------------------
\usepackage{graphicx}
\graphicspath{ {./images} }
\usepackage{wrapfig}
%----------------------

% Import minted
%----------------------
\usepackage{minted}
%----------------------

\linespread{1.3}
\sloppy
\clubpenalty=10000
\widowpenalty=10000



\begin{document}

%--------------------------------------
%			ТИТУЛЬНЫЙ ЛИСТ
%--------------------------------------
\begin{titlepage}
\thispagestyle{empty}
\newpage


%Шапка титульного листа
%--------------------------------------
\vspace*{-60pt}
\hspace{-65pt}
\begin{minipage}{0.3\textwidth}
\hspace*{-20pt}\centering
\includegraphics[width=\textwidth]{emblem}
\end{minipage}
\begin{minipage}{0.67\textwidth}\small \textbf{
\vspace*{-0.7ex}
\hspace*{-6pt}\centerline{Министерство науки и высшего образования Российской Федерации}
\vspace*{-0.7ex}
\centerline{Федеральное государственное бюджетное образовательное учреждение }
\vspace*{-0.7ex}
\centerline{высшего образования}
\vspace*{-0.7ex}
\centerline{<<Московский государственный технический университет}
\vspace*{-0.7ex}
\centerline{имени Н.Э. Баумана}
\vspace*{-0.7ex}
\centerline{(национальный исследовательский университет)>>}
\vspace*{-0.7ex}
\centerline{(МГТУ им. Н.Э. Баумана)}}
\end{minipage}
%--------------------------------------

%Полосы
%--------------------------------------
\vspace{-25pt}
\hspace{-35pt}\rule{\textwidth}{2.3pt}

\vspace*{-20.3pt}
\hspace{-35pt}\rule{\textwidth}{0.4pt}
%--------------------------------------

\vspace{1.5ex}
\hspace{-35pt} \noindent \small ФАКУЛЬТЕТ\hspace{80pt} <<Информатика и системы управления>>

\vspace*{-16pt}
\hspace{47pt}\rule{0.83\textwidth}{0.4pt}

\vspace{0.5ex}
\hspace{-35pt} \noindent \small КАФЕДРА\hspace{50pt} <<Теоретическая информатика и компьютерные технологии>>

\vspace*{-16pt}
\hspace{30pt}\rule{0.866\textwidth}{0.4pt}
  
\vspace{11em}

\begin{center}
\Large {\bf Лабораторная работа № 1} \\ 
\large {\bf по курсу <<Языки и методы программирования>>} \\
\large <<Настройка среды разработки. Запуск простейшего кода программы>> 
\end{center}\normalsize

\vspace{8em}


\begin{flushright}
  {Студент группы ИУ9-22Б Павлов И. П. \hspace*{15pt}\\ 
  \vspace{2ex}
  Преподаватель Посевин Д. П.\hspace*{15pt}}
\end{flushright}

\bigskip

\vfill
 

\begin{center}
\textsl{Москва 2023}
\end{center}
\end{titlepage}
%--------------------------------------
%		КОНЕЦ ТИТУЛЬНОГО ЛИСТА
%--------------------------------------

\newpage
\section{Условие}
В лабораторной работе необходимо выполнить ряд действий для подготовки удобной рабочей среды для написания кода на Java. Дан код класса Factorial, который нужно запустить через терминал и через среду разработки:
\begin{minted}{java}
import java.util.stream.IntStream;

public class Factorial 
{ 

 public static void main(String [] args) 
 { 
  if (args.length == 0) 
     { 
      System.out.println("Usage: java Factorial x"); 
     } 
  else 
     { 
      int n = Integer.parseInt(args [0]);
      var numbers = IntStream.range(1, n+1); 
      int f = numbers.reduce(1, (r,x)-> r*x); 
      System.out.println(f); 
     } 
 } 
} 
\end{minted}

\section{Настройка в терминале}
Была произведена установка Java SDK через терминал. Далее был взят готовый файл с классом факториала и запущен через терминал.
\begin{figure}[h] 
\center{\includegraphics[scale=0.4]{cmd.png}} 
\caption{Вывод в терминале} 
\label{fig:image} 
\end{figure}

\section{Настройка IntelliJ IDEA}
Была произведена установка IntelliJ через официальный сайт разработчика. Был создан проект с классом факториала внутри.
\begin{figure}[h] 
\center{\includegraphics[scale=0.35]{intellij_cmd.png}} 
\caption{Вывод в терминале IntelliJ} 
\label{fig:image} 
\end{figure}
\begin{figure}[h] 
\center{\includegraphics[scale=0.35]{intellij_run.png}} 
\caption{Стандартный вывод IntelliJ} 
\label{fig:image} 
\end{figure}

\end{document}
